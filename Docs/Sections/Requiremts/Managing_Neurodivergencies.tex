This section is solely dedicated to Asmaa Al-Hajeri's project, and is not directly connected with the project
described above.

\subsection{App description:}\

	 A productivity app that incorporates gaming into everyday tasks. A character is custom created by the
	 user, the user then goes on to add their to-do list with all their tasks (different to-do lists are
	 provided if the user works and does chores) After the user chooses a task to start, the character starts
	 to work on a task of their own (can be an artwork that’s shown at the end of the day or something else)
	 Once the user finishes their task (after a certain amount of time or manually) they check the task as
	 “done!” then the user is able to spend their break listening to music with their character, check on
	 their character’s plants, make a meal with their character (recipes can be provided if needed) Then the
	 cycle repeats until all the tasks are completed. The user is also allowed to create a certain schedule,
	 meaning that certain tasks are repeated every time the user wants it to be (users can add their own
	 recipes to be repeated on certain days of the week, they can also add a playlist for them to listen to
	 on their breaks, they can add facts they know about the pets/plants they own so when the online feature
	 is added, other people also know these facts).

\subsection{How does it help the community:}\

	The target demographic is neurodivergent people. Due to their struggles of finding a good incentive to
	start and finish tasks, this app rewards them after every task while their in-game character takes a
	break with them.

\subsection{Algorithm:}

	\paragraph{Main algorithm:}
		\begin{enumerate}

			\item Use the Scanner class to get the input from the user

			\item Use the while loop to keep taking input from the user when the user wants to add a task,
			add a
			fact, or add a recipe

			\item If the user chooses to add a recipe, the user is asked if they would like to categorize it
			as a
			“breakfast”, “lunch”, “dinner” meal, or a “snack”. Otherwise, store the recipe as a “meal”

			\item Once the user chooses to add a recipe, the user is asked to submit the number of steps it
			takes
			to complete the recipe and then is asked to enter the steps based on the number assigned

			\item If the user chooses to add a task, the user is asked if the task is “work” related or
			“home”
			related or “self” related, work tasks include “respond to emails” and “submit assignments”, home
			related tasks include “do the dishes” and “cook”, self related tasks include “read” and “watch a
			movie”

			\item Once the user chooses the category the task belongs to, the user is asked how much time it
			would take for them to complete it. If the user sets an approximation, the user will be reminded
			about the task after the time for that task is over, otherwise the task will just be set without
			an
			approximate time of its completion

			\item The user is then asked to choose how long the breaks between the tasks to be

			\item Use the for loop to keep taking input from the user to take their to do list

			\item Use Math.random and assign it to a bundle of welcoming messages for when the user first
			enters
			the app in the beginning of day

			\item Use ArrayList in order to put a limit to the amount of characters written in a task

			\item Create a method that stops the task from looping if the user enters “Done”

			\item Set a timer that starts once step 10 is completed

			\item The timer is set based on how long the user chose the break between the tasks to be.

			\item Stop the timer once the set amount of time has passed and print the second task in the
			To-Do
			list

		\end{enumerate}

	\paragraph{Side  algorithm:}
		\begin{enumerate}
			\item Use Array to encrypt messages for the user to decrypt and complete a quest

			\item Use switch, case, and Array to create a mini TicTacToe game as a quest

			\item Declare a plant object that can be planted by the user and grows over time

			\item Declare a cat object that can play with the user and requires to be fed

			\item Create an object that outputs random vocabulary and a Scanner object that takes the input
			of the user (word association)

			\item Create a method that creates a randomised order of “verb,” “adjective,” “proper noun,”
			“common noun,” and “adverb” in a story where the user enters the wanted sentence part to complete
			the story

			\item Create a URL object that and a URL method that accepts a hyperlink from music players set
			by the user

			\item Create method that calculates the average amount of steps taken in the set amount of time

		\end{enumerate}

\subsection{How to improve it in the future:}\

		Add an online feature where people could visit other people’s places and leave notes, water their
		plants, etc.
