\paragraph{Requirements:}\

	The user interface of the VHMS will be split into three parts: patient, Doctor, and
	master interface. Each interface will be further broken up into parts, see fig
	\ref{figure:architect:modules} for how UI will be split
	up.


	\begin{figure}[H]
		\centering
		\includegraphics[scale=0.20]{Sections/Requiremts/pics/UI digram.drawio.png}
		\caption{VHMS UI outline}
		\label{figure:architect:modules}
	\end{figure}

\subparagraph{Patient Interface}\

	As the figure above shows, the patient interface will be split into three
	main
	parts:

	\begin{itemize}
		\item Patient Information: This section will consist of all information attainable in regard to the
		patient. For example, things such as medication the patient might be on, chronic conditions, medical
		records(such as x-rays, blood tests, etc.), and any other pieces of information that would be
		relevant to the doctor to make medical evaluations.\label{Patient-information1}

		\item Booking Options: A booking page will be available for the patient. This page will show
		patients/visitors what times they can come in for an appointment.

		\item Billing/Visit History: Patients will have access to a timeline of visits. This timeline will
		show information such as: purpose of visit, any medication prescribed in the visit, and an itemized
		list of all things the patient is billed for in the visit.
	\end{itemize}

\subparagraph{Master Interface}\

	The master interface is the most expansive of the three interfaces, it will consist of five major parts:

\begin{itemize}
	\item Patient Information: Patient information in the master will differ from the one available to
	patients. The master patient information will show a list of all patients where all the information
	(described in patient interface \ref{Patient-information1}) can be accessed and
	edited.\label{Patient-information2}

	\item  Booking Information: In this page persons with access to the master can see all bookings, edit
	them, and assign new booking if need be.

	\item Employee information: This page will host a lost similar to that of patient information page, but,
	it will have information regarding employees, such as: static info like; name, role, rank, etc.) and more
	'dynamic' info like: patients and wards the employee is assigned to, tools and equipment used or is being
	used by the employee, etc.

	\item Stock: Stocks will be list of all items needed for running the hospital. stocks will be split into
	few parts, such as: medication, medical supplies, cleaning supplies, beds and rooms will fall under
	stock, and other miscellaneous items that are need for the running of the hospital. further details about
	the stock system can be found in the stock section found in \ref{Stock}

	\item Master log: The log will be record of all changes happening in the software. Any edits, additions,
	removals of anything will be seen in the log with the time and file edited listed.
\end{itemize}


\subparagraph{Doctor Interface}\

	The doctor interface will display the same info as patient information interface with some of the edit
	access of the master, and will consist of three parts:
\begin{itemize}
	\item Patient information: the doctors Patient information page will be similar to that of the master(
	descried above in \ref{Patient-information2}), with notable exception that the doctor will only have
	access to the patients that they have an appointment with or ones that are assigned to them.

	\item  incoming bookings: This page will show the doctor all his appointments and with which patients.

	\item doctor documents: doctor documents will consist of paper work the doctor might want to access, such
	as prescribing x-rays, MRIs, blood tests or other producers that require paper work.
\end{itemize}



\paragraph{Preliminary Study:}\

The current consideration is to build the ui in either SceneBuilder, with JavaFX libary, or  Codename One.
further investigation in each programs needs to be made before a final decisions is made in which program to
build in.
